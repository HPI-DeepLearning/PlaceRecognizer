\section{Outlook}
We have several ideas to improve our application.\\

Right now the application is very basic. It only contains two buttons which call the most essential functions and show the most essential information. Improvements of the visualization, like showing the camera image in full screen and switching to the information pane on getting a result, providing an option page, for example for switching language or user or possibilities to receive further information about a recognized place by showing links to Wikipedia or Google could add a lot of user experience to the application.\\
Because our application is aimed at international tourists, it is a good idea to localize it. This can be achieved by providing translation resource files, which cold look like this:
\begin{lstlisting}[language=XML, basicstyle=\scriptsize]
    #/values/strings.xml:
    <resources>
        <string name="hello_world">Hello World!</string>
    </resources>
\end{lstlisting}

\begin{lstlisting}[language=XML, basicstyle=\scriptsize]
    #/values-es/strings.xml
    <resources>
        <string name="hello_world">iHola Mundo!</string>
    </resources>
\end{lstlisting}

Also for the Wikipedia parser, we can get the phone language with a simple line of code
\begin{lstlisting}
    Locale.getDefault().getDisplayLanguage();
\end{lstlisting}

Lastly, a naive solution for improving the recognition results, especially if many objects are nearby is pre-filtering the possible results using the current position and only allowing results that are within a certain radius. A more complex approach would be to include the coordinates into the training itself. That means that a result is not only based on computing pixel values, but also considers the current position. This would provide an additional degree of confidence into the calculated result.\\
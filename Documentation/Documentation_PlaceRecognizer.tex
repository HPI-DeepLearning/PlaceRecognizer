% Koma-Script Basisklasse
\documentclass[a4paper,12pt,pagesize,headsepline,bibtotoc,titlepage]{scrartcl}

\usepackage[english]{babel}
\usepackage[utf8]{inputenc}     % direkte Eingabe von Umlauten & Co. (Vorsicht: Encoding im Editor muss auch UTF-8 sein!)

\usepackage[T1]{fontenc}            % T1-Schriften

\usepackage{mathptmx}           % Times/Mathe \rmdefault
\usepackage[scaled=.90]{helvet} % Skalierte Helvetica \sfdefault
\usepackage{courier}            % Courier \ttdefault

% Zusatzpakete für mehr mathematische Symbole, Einfügen von Grafiken
% und bessere Bildunterschriften
\usepackage{amsmath,amsthm,amsfonts,graphicx,caption}

% Wenn man direkt mit dem pdflatex eine PDF-Datei erzeugt, sollten diese beiden Pakete eingebunden werden
\usepackage{hyperref} % Hyperlinks anklickbar
\usepackage{ae,aecompl} % bessere Bildschirmschriftarten usw.
\usepackage{cite}
\usepackage{listings}
\usepackage{float}

\usepackage[dvipsnames]{xcolor}
\usepackage{textcomp}
\lstdefinelanguage{XML}
{
  basicstyle=\ttfamily\footnotesize,
  morestring=[b]",
  moredelim=[s][\bfseries\color{Maroon}]{<}{\ },
  moredelim=[s][\bfseries\color{Maroon}]{</}{>},
  moredelim=[l][\bfseries\color{Maroon}]{/>},
  moredelim=[l][\bfseries\color{Maroon}]{>},
  morecomment=[s]{<?}{?>},
  morecomment=[s]{<!--}{-->},
  commentstyle=\color{DarkOliveGreen},
  stringstyle=\color{blue},
  identifierstyle=\color{red}
}
\lstdefinelanguage{Java}
{
  showspaces=false,
  showtabs=false,
  breaklines=true,
  showstringspaces=false,
  breakatwhitespace=true,
  commentstyle=\color{pgreen},
  keywordstyle=\color{pblue},
  stringstyle=\color{pred},
  basicstyle=\ttfamily,
  moredelim=[il][\textcolor{pgrey}]{$$},
  moredelim=[is][\textcolor{pgrey}]{\%\%}{\%\%}
}

\pagestyle{headings}

% Abstand der Kopfzeile vom Text:
\headsep4mm

\typearea[current]{current}     % Satzspiegel neu berechnen

% andere Bildunterschrift mit Hilfe von caption
\renewcommand{\figurename}{Abb.}
\renewcommand{\captionlabelfont}{\bf}

\title{
    \includegraphics*[width=0.4\textwidth]{hpi_logo.png}\\
    \vspace{24pt}
    Recognizing Famous Places on Android
}
\subtitle{
    Seminar\\
    Practical Applications of Multimedia Retrieval\\
    Winter Semester 2016/2017
}
\author{
    Tim Oesterreich, Romain Granger\\[12pt]
    Supervisors:\\
    Haojin Yang, Christian Bartz\\
    Prof. Dr. Christoph Meinel
}
\date{\today}

\begin{document}
\lstset{language=Java}
\maketitle
\tableofcontents
\newpage

\section{Introduction}
As deep learning is becoming an increasingly big influence in everyday applications, more and more focus is put into increasing its distribution to different platforms.\\
During the winter semester 2016/2017 a project seminar emerged, that laid focus on developing a mobile phone application for Google's Android operating system, that is capable of recognizing famous sights in big cities, like Berlin, from images taken with a smartphone camera.\\
Modern smartphones can, in some cases, outperform mid-range notebooks from a couple of years ago and, most interestingly, often have a dedicated GPU\footnote{Graphics Processing Unit}. GPUs are usually used for deep learning because of their high parallelization capabilities.\\
Part of this seminar was the evaluation of a fairly recent deep learning framework called CNNDroid \cite{cnndroid2016}, which uses GPU acceleration for classification and promises substantial performance improvements compared to CPU classification.
\section{CNNDroid}
CNNDroid is an open source deep learning library for Android. It is able to execute convolutional neural networks, supporting most CNN layers used by existing desktop/server deep learning frameworks, namely Caffe, Theano and Torch. Supported layers, as of March 2017 are:
\begin{itemize}
    \item{Convolutional Layer}
    \item{Pooling Layer}
    \item{Local Response Normalization Layer}
    \item{Fully-Connected Layer}
    \item{Rectified Linear Unit Layer (ReLU)}
    \item{Softmax Layer}
    \item{Accuracy and Top-K Layer}
\end{itemize}
Due to the library being open source, it is possible to add additional layers, such as batch normalization or sigmoid.\\
The library also supports a variety of customizations, like maximum memory usage, GPU or CPU acceleration and automatic performance tuning.

\subsection{Setup and Integration into Android Project}
CNNDroid is a source code library. That means that integration into an existing Android Project is fairly straight forward and doesn't require any third-party dependencies.\\
The only prerequisites are:
\begin{itemize}
    \item{A functional Android development environment (e.g. Android Studio\footnote{https://developer.android.com/studio/index.html})}
    \item{Android phone or emulator running at least Android SDK version 21.0 (Lollipop)}
\end{itemize}

To integrate CNNDroid into the project, the repository has to be cloned from its GitHub\footnote{https://github.com/ENCP/CNNdroid} page. Inside the \texttt{CNNdroid Source Package}' folder are three folders: \texttt{java}, \texttt{rs} and \texttt{libs}.\\
The \texttt{java} and \texttt{rs} folders need to be copied into the projects \texttt{app/src/main/} directory, merging the \texttt{java} folders.\\
The \texttt{libs} folder has to be copied and merged into the \texttt{app/} directory.\\
For effective usage of CNNDroid, it needs read and write access to storage on the smartphone. Android policies require an application to request these permissions before the app starts. These permission requests are specified inside the \texttt{AndroidManifest.xml} file. The following lines have to be added there before the \lstinline[language=XML]{<application} section:

\begin{lstlisting}[language=XML, basicstyle=\scriptsize]
    <uses-permission android:name="android.permission.READ_EXTERNAL_STORAGE"/>
    <uses-permission android:name="android.permission.WRITE_EXTERNAL_STORAGE"/>
\end{lstlisting}

To cope with the high computational requirements, CNNDroid potentially needs a large amount of memory. Therefore, the heap has to be increased, which is done by adding

\begin{lstlisting}[language=XML, basicstyle=\scriptsize]
    android::largeHeap="true"
\end{lstlisting}
\noindent
inside the \lstinline[language=XML]{<application} section.\\
Now, everything should be set up correctly and programming can commence.\\
After importing the package \lstinline[language=Java]{network.CNNdroid}, the \lstinline[language=Java]{CNNdroid} object is accessible, which can call the function \lstinline[language=Java]{CNNDroid::classify}. This is the keyword to start execution of the trained model. The specification of the model is explained in the following section.

\subsection{Structure of necessary CNNDroid Files}
Of course, CNNDroid needs to know how to classify an input. For that it either uses converted binary layer files, created by a desktop deep learning framework, or internal conversion functions (e.g. for pooling or ReLU).\\
The layer BLObs\footnote{Binary Large Objects} are generally used for more complicated layers, such as convolutional or fully connected layers, which have high dimensional variables (weight matrices). These files have to be converted into the MessagePack\footnote{http://msgpack.org/} format and put onto the smartphones storage.\\
The order in which the layers are executed, additional configurations and layers which do not need a BLOb file are defined inside the definition file. This file is used as a settings file for the \lstinline[language=Java]{CNNdroid} classifier.\\
On creation of the \lstinline[language=Java]{CNNdroid} object, the absolute path to this file has to be specified.\\
Configurations, other than the layer definitions are:
\begin{itemize}
    \item{the absolute path to the layer blob files\\(\lstinline[language=Java]{root_directory: "/absolute/path/to/layer/files"})}
    \item{the maximum amount of RAM the classifier should use on the smartphone\\(\lstinline[language=Java]{allocated_ram: Amount_in_MB})}
    \item{whether or not automatic performance tuning should be used\\(\lstinline[language=Java]{auto_tuning: "on|off"})}
    \item{whether or not classification should be GPU accelerated (if possible)\\(\lstinline[language=Java]{execution_mode: "sequential|parallel"})}
\end{itemize}
Layers are defined as shown in figure \ref{fig:def_file}.

\begin{figure}[H]
  \centering
    \includegraphics[width=0.5\textwidth]{def_file.png}
  \caption{Excerpt of a CNNDroid definition file}
  \label{fig:def_file}
\end{figure}

Not essential for the execution of CNNDroid, but very helpful for processing the final results, is a labels file. This file should specify human-readable names for the extractable classes. The order inside this file should be mapped to the output order of the last layer (probably a fully-connected layer), i.e. the first value of the output array should correspond to the first class name inside the labels file. The class names should be newline-separated.

\subsection{Training the model}
We trained our deep learning model based on crawled images with the caffe framework. The training occurred on an HPI server (accessible under the IP 172.16.18.178 and username msws2016t1). Caffe is already installed there and a variety of crawled images, as well as a labels file with the paths to each image are available inside the \texttt{training\_data} folder. Inside the same folder you can also find the \texttt{training\_full.sh} script, which can be executed to start the training and create a model.\\
An example model can be found in the \texttt{training\_data/berlin\_sights\_model} folder.\\
In order to test the model quickly, caffe provides an ipython notebook on their Github repository. This repository is checked out in the \texttt{caffe} folder. To start the notebook an ssh-tunnel has to be created to your local machine, e.g. using\\\\
\texttt{ssh -N -f -L localhost:8888:localhost:8889 msws2016t1@172.16.18.178}\\\\
Using a normal ssh connection to the server, the notebook can be executed with the command\\\\
\texttt{ipython notebook --no-browser --port=8889}\\\\
and viewed in a webbrowser on the local machine under the address \texttt{localhost:8888}.\\
The notebook itself provides possibilities to classify an image downloaded from a web address, view the output of specific layers and more.

\subsection{Convert Trained Models into CNNDroid-compatible Format}
As mentioned before, CNNDroid uses a format called MessagePack for the layer definitions. MessagePack is a binary serialization format for exchanging data, comparable to JSON\footnote{JavaScript Object Notation}, but other than JSON it is not human-readable. The omission of this constraint makes it possible to reduce size and increase parsing speed of the data stored inside. This is especially important due to the limited storage that most smartphones posses, especially compared to big servers, where deep learning is usually executed.\\
Figure \ref{fig:json_vs_msgpack} illustrates how MessagePack saves storage space by using types (JSON encodes everything as strings, so for example, `true' uses four Bytes, while it only uses one Byte using MessagePack) and dropping control symbols, like the curly braces \{\}.

\begin{figure}[H]
  \centering
    \includegraphics[width=0.5\textwidth]{json_vs_msgpack.png}
  \caption{Comparison JSON and MsgPack}
  \label{fig:json_vs_msgpack}
\end{figure}

In order to convert models from the aforementioned desktop/server deep learning frameworks, CNNDroid provides conversion scripts, which can be found in the\\
\texttt{CNNDroid/Parameter Generation Scripts/} folder. The script for Caffe is written in Python and requires the packages \lstinline[language=python]{numpy} and \lstinline[language=python]{msgpack} to be installed.\\
Three variables need to be defined.
\begin{itemize}
  \item{\texttt{MODEL\_FILE} - The absolute path to the trained model}
  \item{\texttt{MODEL\_NET} - The absolute path the deployment prototxt file, which specifies the parameters for the layers}
  \item{\texttt{SAVE\_TO} - The saving path}
\end{itemize}
The output are converted layer files in MessagePack format. The definition and label files have to be created manually. For the definition file, most of the content of the prototxt deployment file can be used. Only some parts of the syntax have to be adapted to a CNNDroid parsable format.\\
\newpage
\subsection{CPU vs GPU performance}
For a comparison of the sequential and parallel mode of CNNDroid we used the Android emulator using 2 GB of RAM. We used the berlin sights classifier, provided in the berlin\_sights\_data folder.

\begin{center}
  \begin{tabular}{l|c|r}
    layer & sequential time in ms & parallel time in ms \\\hline
    conv1 & 277 & 75\\
    pool1 & 22 & 47\\
    relu1 & 1 & 1\\
    conv2 & 299 & 27\\
    pool2 & 8 & 23\\
    conv3 & 161 & 24\\
    pool3 & 4 & 6\\
    ip1 & 1 & 5\\
    ip2 & 0 & 0\\
    prob & 0 & 0\\\hline
    sum & 773 & 208\\
  \end{tabular}
\end{center}

We can see that especially the more complicated convolutional layers profit from the additional speed of the GPU. Copying data from the RAM to GPU memory is usually very high, which is why the sequential mode can be faster for layers that do not do a lot of processing. In general, the parallel, GPU-accelerated processing speed is about three times faster than sequential, CPU-only processing.\\
The CNNDroid paper\cite{cnndroid2016} states a 12-times speedup with the CIFAR10 dataset on a Samsung Galaxy Note 4.\\
Our test machine was a MacBook Pro with an Intel Iris Graphics 6100 GPU and a 2.9 GHz Intel Core i5 CPU using the Android Studio emulator.\\
Apparently, the emulator does not reach such a significant overall speedup as the one stated in the paper. The \texttt{conv2} layer reaches an 11-times speedup, which comes close, but in sum it only reaches a third of the expected performance.\\
This however can potentially have a variety of reasons. Our testing setup does not follow lab conditions. It is not possible to know how the Android Studio emulator uses the MacBooks resources. Additional overhead is to be expected when communicating with the hardware. Also the CPU in the MacBook is more powerful than the one inside the Samsung phone (using an NVidia Jetson-style dual CPU consisting of one Quad-core 1.9 GHz Cortex-A57 and one Quad-core 1.3 GHz Cortex-A53, usually only using the A57 for high-computation tasks), which would decrease the speedup for the parallelized computation as the sequential computation is faster.\\
In theory the GPU of the MacBook should perform a lot better than the Samsung Galaxys GPU (which is capable of 48 parallel tasks, while the MacBooks GPU is capable of many hundred computations at the same time), it is however possible that the layer doesn't provide so many calculations, so that the full performance can not be used.\\
Overall, it is apparent, that the parallelization of the classification creates a clear speedup, which does however not reach that promised in the paper using our test setup.
\section{Google Streetview Crawler}
% Don't know yet if I need subsections for these
\subsection{Setup of viewing parameters}
Explain csv file

\subsection{State of Automation (i.e. taking one image per viewing angle)}

\subsection{Current Limitations}
Full automation not possible as of current street view API state; wrong latitude/longitude; Streetview Image API returns different images than JavaScript API (which is being used on google maps website)

\subsection{Possible Improvements}
Use Classifier to identify things in photsphere

\subsection{Google Places API/shutterstock/Flickr}
\section {PlaceRecognizer Application}
After crawling the images and training a model, this section explains how we used all these prerequisites to create an Android application that is capable of classifying images taken with the camera.

\subsection {CNNDroid Integration/Image Classifier}
Explain ImageClassifier Class; Including Variables that need adaption when changing Layers or DataSets\\
How to put msgpack on phone

\subsection {Real-Time Frame Capture}
How does the camera talk to the Image Classifier?

\subsection {GPS Logger}
How do we get GPS values and how can we integrate them?

\subsection {Wikipedia Parser}
When our CNNDroid model has come with an image class, we now want to get informations about the place. Wikipedia represent one of the largest source of informations and has a big and active community which update regularly all the informations. Also, the list of features we can grab from wikipedia is long: Descriptions, images, literature etc..
Also, compared to other API in a technical way, Wikipedia API present three mains advantages:
\begin{itemize}
    \item{The use is made by a simple url call}
    \item{There use no use limitations or constraint}
    \item{It provides a Json response easily parsable by any program. }
\end{itemize}
The wikipedia API is composed by a principal URL which is "https://en.wikipedia.org/w/api.php".Then you can easily change the language of the response by changing the subdomain of the URL. For exemple to get a french Json content, you can call "https://fr.wikipedia.org/w/api.php". Like our application will be useful for tourists, we need to adapt our content to their language.
Then, the principal URL will be enrich with paramters, like the "format" (xml,json,html) or the "titles" to get informations about a specific page. We will see in the second part of this section which parameters we use to call the wikipedia API.\\\\
In our program we use two classes:
\begin{itemize}
    \item{HttpHandler.java}
\end{itemize}
The first class, "HttpHandler.java" is here to create a temporary array of bytes from the url and also catch error to check if it is correct and if it returns a value. In a second time it will convert the response stream as a string value.
So first, we aim to use the http GET method to open a connection.
\begin{lstlisting}[language=XML, basicstyle=\scriptsize]
try {
    URL url = new URL(reqUrl);
    HttpURLConnection conn = (HttpURLConnection)
    url.openConnection();
    conn.setRequestMethod("GET");
\end{lstlisting}
Then we will catch the errors, for the following cases:Incorrect input url, Protocol exception, Input/Output error during the use of the InputStream() function, or if the response is an empty array of bytes.
\begin{lstlisting}[language=XML, basicstyle=\scriptsize]
    } catch (MalformedURLException e) {
            Log.e(TAG, "MalformedURLException: " + e.getMessage());
        } catch (ProtocolException e) {
            Log.e(TAG, "ProtocolException: " + e.getMessage());
        } catch (IOException e) {
            Log.e(TAG, "IOException: " + e.getMessage());
        } catch (Exception e) {
            Log.e(TAG, "Exception: " + e.getMessage());
\end{lstlisting}

\begin{itemize}
    \item{GetWiki.java}
\end{itemize}
This class is a public class using an asynchronous task. In fact, our classifier will give a class to the image taken by the user, for exemple "Brandenburg Gate" or "Fernsehturm Berlin". Then this label will be used as the input string variable for our GetWiki class. First, the call is made from the mainActivity class, with the following statement:
\begin{lstlisting}[language=XML, basicstyle=\scriptsize]
wikipediaInfos = new GetWiki().execute("classLabel")
\end{lstlisting}
"classLabel" is a string variable, which is dynamically update with the class given by the classifier. So it is also important to give a reliable class label regarding wikipedia when the model is trained, because this label will be used as a parameter in the url which is called:
\begin{lstlisting}[language=XML, basicstyle=\scriptsize]
String urlTitle = strings[0];
String url = "https://en.wikipedia.org/w/api.php?format=json&action=query&prop=extracts&exintro=&explaintext=&titles=" + urlTitle;
\end{lstlisting}
The result will consist in a Json array that we need to parse to extract part of the information we want. An expemple of response look like the following:
\begin{lstlisting}[language=XML, basicstyle=\scriptsize]
    {"batchcomplete":"","query":{"pages":{"156604":{"pageid":156604,"ns":0,"title":"Brandenburg Gate","extract":"The Brandenburg Gate (German: Brandenburger Tor) is an 18th-century neoclassical monument in Berlin, and one of the best-known landmarks of Germany. It is built on the site of a former city gate that marked the start of the road from Berlin to the town of Brandenburg an der Havel....\n"}}}}
\end{lstlisting}
We truncated the response to keep only the variable we actually use like the title and the description. As you can the response give us several Json levels/objects. We will use the JsonObject package given in the Android API 25, to parse our Json response. In our example, we would like to get all informations contained the 156606:{ object. The problem is that we can't guess the Id of the page, so we will do like this:
\begin{lstlisting}[language=XML, basicstyle=\scriptsize]
JSONObject jsonObj = new JSONObject(jsonStr);
JSONObject query = jsonObj.getJSONObject("query");
JSONObject pages = query.getJSONObject("pages");
Iterator<String> keys = pages.keys();
String pageId= keys.next();
JSONObject page = pages.getJSONObject(pageId);
String title = page.getString("title");
String extract = page.getString("extract");
\end{lstlisting}
First, we get the response into a JsonObject class variable. Then we can use string parameters to navigate through the Json file. If we have an object that is not static, like the Id of the page, we use an Iterator keys which will return the value for this Json Object, here it will give us the pageId (156606) as a string. Then we just have to say that we want to go to the next value of the json array using keys.next().\\
Finally, we get the string value of the response and store them into variables.

\subsection{Text to speech}
To enhance the user experience of our app, as it is used to recognise famous places, we had a look on how to allow the user to play an audio description of the place while is watching it. Android SDK provide a very useful package:
\begin{lstlisting}[language=XML, basicstyle=\scriptsize]
import android.speech.tts.TextToSpeech
\end{lstlisting}
which has built in methods to parse a string value and read it as an audio track. It can easily be implemented with few lines, following these steps:\\\\
First, you need to declare a TTS (text to speech) object, and set the attributes. Let's say in our exemple that "t1" is our object. Many attributes can be changed, like the languages, the speed of the audio, the voice type.We declare in the following snippet the variable 't1' as a new text to speech object in our class context, giving the langage Locale.UK
\begin{lstlisting}[language=XML, basicstyle=\scriptsize]
t1=new TextToSpeech(getApplicationContext(), new TextToSpeech.OnInitListener() {
            @Override
            public void onInit(int status) {
                if(status != TextToSpeech.ERROR) {
                    t1.setLanguage(Locale.UK);
                }
            }
        });
\end{lstlisting}
Then, you need to get the string variable you want to convert as an audio file. In the following snippet we get the wikipedia description of Branderbuge Gate from our wikipedia API parser (see the previous part)
\begin{lstlisting}[language=XML, basicstyle=\scriptsize]
String toSpeak = null;
try {
toSpeak = new GetWiki().execute("Brandenburg_Gate").get().descritpion;
    } catch (InterruptedException e) {
      e.printStackTrace();
    } catch (ExecutionException e) {
      e.printStackTrace();
    }
\end{lstlisting}

Finally, you can use the speak() method from your object, with the queue mode (QUEUE FLUSH) which means that media to be played are dropped and replaced by the new entry each time you call this method.

\begin{lstlisting}[language=XML, basicstyle=\scriptsize]
t1.speak(toSpeak, TextToSpeech.QUEUE_FLUSH, null , "test");
\end{lstlisting}

\subsection{Firebase API}
Firebase is a Google service which provide a realtime database and backend as a service. The service is really well integrated with Android.  The data is stored on Firebase's cloud and the Firebase SDK for android comes up with several method to store and synchronised in real time all the data.

\begin{itemize}
    \item {How we use it}
\end{itemize}

First, we declare in the build.gradle at the app level the use of Firebase service.
\begin{lstlisting}[language=XML, basicstyle=\scriptsize]
compile 'com.google.firebase:firebase-core:10.0.1'
compile 'com.google.firebase:firebase-database:10.0.1'
\end{lstlisting}

Then, like the GooglePlaceAPI, we need to settle an object of type database to start the connection with our database instance.

\begin{lstlisting}[language=XML, basicstyle=\scriptsize]
private FirebaseAuth mAuth;
private FirebaseAuth.AuthStateListener mAuthListener;
DatabaseReference database = FirebaseDatabase.getInstance().getReference();
\end{lstlisting}

The Firebase structure use path to store the data. For exemple, you will have the top level path "users", then each users of you app, and for each users, some features. To store or read data, the method will look like this:

\begin{lstlisting}[language=XML, basicstyle=\scriptsize]
database.child("users").child(userId).child("places").push("location");

\end{lstlisting}
You need to define all the levels you want to have access to. Here we first access the "users" node, then getting the current user ID, and we upload the locations details of his visit into the "places" node.

\subsection{How to setup project in Android Studio}
The project should work without any additional work in Android Studio. Simply import it into the IDE, setup an Android emulator or plug in an android phone running at least Android SDK 24.0 and press the ``Run'' button.
\section{Outlook}
We have several ideas to improve our application.\\

Right now the application is very basic. It only contains two buttons which call the most essential functions and show the most essential information. Improvements of the visualization, like showing the camera image in full screen and switching to the information pane on getting a result, providing an option page, for example for switching language or user or possibilities to receive further information about a recognized place by showing links to Wikipedia or Google could add a lot of user experience to the application.\\
Because our application is aimed at international tourists, it is a good idea to localize it. This can be achieved by providing translation resource files, which cold look like this:
\begin{lstlisting}[language=XML, basicstyle=\scriptsize]
    #/values/strings.xml:
    <resources>
        <string name="hello_world">Hello World!</string>
    </resources>
\end{lstlisting}

\begin{lstlisting}[language=XML, basicstyle=\scriptsize]
    #/values-es/strings.xml
    <resources>
        <string name="hello_world">iHola Mundo!</string>
    </resources>
\end{lstlisting}

Also for the Wikipedia parser, we can get the phone language with a simple line of code
\begin{lstlisting}
    Locale.getDefault().getDisplayLanguage();
\end{lstlisting}

Lastly, a naive solution for improving the recognition results, especially if many objects are nearby is pre-filtering the possible results using the current position and only allowing results that are within a certain radius. A more complex approach would be to include the coordinates into the training itself. That means that a result is not only based on computing pixel values, but also considers the current position. This would provide an additional degree of confidence into the calculated result.\\

\newpage
\bibliography{bibliography}{}
\bibliographystyle{plain}

\end{document}
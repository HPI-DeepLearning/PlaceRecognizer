% Koma-Script Basisklasse
\documentclass[a4paper,12pt,pagesize,headsepline,bibtotoc,titlepage]{scrartcl}

\usepackage[ngerman]{babel}     % deutsche Trennmuster
\usepackage[utf8]{inputenc}     % direkte Eingabe von Umlauten & Co. (Vorsicht: Encoding im Editor muss auch UTF-8 sein!)

\usepackage[T1]{fontenc}            % T1-Schriften

\usepackage{mathptmx}           % Times/Mathe \rmdefault
\usepackage[scaled=.90]{helvet} % Skalierte Helvetica \sfdefault
\usepackage{courier}            % Courier \ttdefault

% Zusatzpakete für mehr mathematische Symbole, Einfügen von Grafiken
% und bessere Bildunterschriften
\usepackage{amsmath,amsthm,amsfonts,graphicx,caption}

% Wenn man direkt mit dem pdflatex eine PDF-Datei erzeugt, sollten diese beiden Pakete eingebunden werden
\usepackage{hyperref} % Hyperlinks anklickbar
\usepackage{ae,aecompl} % bessere Bildschirmschriftarten usw.

\pagestyle{headings}

% Abstand der Kopfzeile vom Text:
\headsep4mm

\typearea[current]{current}     % Satzspiegel neu berechnen

% andere Bildunterschrift mit Hilfe von caption
\renewcommand{\figurename}{Abb.}
\renewcommand{\captionlabelfont}{\bf}

\title{
    \includegraphics*[width=0.4\textwidth]{hpi_logo.png}\\
    \vspace{24pt}
    Recognizing Famous Places on Android
}
\subtitle{
    Seminar\\
    Practical Applications of Multimedia Retrieval\\
    Winter Semester 2016/2017
}
\author{
    Tim Oesterreich, Romain Granger\\[12pt]
    Supervisors:\\
    Haojin Yang, Christian Bartz\\
    Prof. Dr. Christoph Meinel
}
\date{\today}

\begin{document}
\maketitle
\tableofcontents
\newpage

\section{Introduction}
\section{CNNDroid}
What is CNNDroid

\subsection{Setup and Integration into Android Project}
\begin{itemize}
    \item{Clone From Github}
    \item{Copy Files into Android Project}
    \item{Necessary Import: import network.CNNdroid}
    \item{Create CNNDroid Object}
    \item{Call CNNDroid::classify}
\end{itemize}

\subsection{Structure Overview of necessary CNNDroid Files}
\begin{itemize}
    \item{Layer Blob Files}
    \item{Definition File}
    \item{Labels}
\end{itemize}

\subsection{Convert Trained Models into CNNDroid-compatible Format}
Short Introduction to MessagePack
\begin{itemize}
    \item{how to use convertion script}
    \item{compatible frameworks}
    \item{compatible layers}
\end{itemize}

\subsection{CPU vs GPU performance}
Comparison of computation time of CPU (sequential) and GPU (parallel) mode on in-memory images using CIFAR10 (in-memory to minimize error using camera, CIFAR10 could be exchanged with CaffeNet if too fast)

\newpage
\section{Google Streetview Crawler}
% Don't know yet if I need subsections for these
\subsection{Setup of viewing parameters}
Explain csv file

\subsection{State of Automation (i.e. taking one image per viewing angle)}

\subsection{Current Limitations}
Full automation not possible as of current street view API state; wrong latitude/longitude; Streetview Image API returns different images than JavaScript API (which is being used on google maps website)

\subsection{Possible Improvements}
Use Classifier to identify things in photsphere

\subsections{Google Places API/shutterstock/Flickr}

\newpage
\section {PlaceRecognizer Application}
Overview: What does it do. Whom is it for. How does it achieve its task?

\subsection {CNNDroid Integration/Image Classifier}
Explain ImageClassifier Class; Including Variables that need adaption when changing Layers or DataSets\\
How to put msgpack on phone

\subsection {Real-Time Frame Capture}
How does the camera talk to the Image Classifier?

\subsection {GPS Logger}
How do we get GPS values and how can we integrate them?

\subsection {Wikipedia Parser}
How do we get the text for a classified image from Wikipedia?

\subsection{How to setup project in Android Studio}

\section{Outlook}


\begin{figure}[hbp]
\begin{center}
\includegraphics*[width=0.75\textwidth]{beispiel.png}\\
\caption{Eine Abbildung, Quelle: \cite{willems2008}}
\label{abb:test}
\end{center}
\end{figure}

\newpage
\begin{thebibliography}{1}

\bibitem{willems2008}
C.~Willems and C.~Meinel.
``Tele-Lab IT-Security: an Architecture for an online virtual IT Security Lab'',
\emph{International Journal of Online Engineering (iJOE)},
X, 2008.

\end{thebibliography}
\end{document}
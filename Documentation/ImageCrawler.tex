\section{Google Streetview Crawler}
Classification of real-world images requires a very big dataset for the learning process beforehand. Google Streetview provides 360 degree images from many places in all German urban areas. Using sophisticated crawling techniques, a huge amount of image data can be extracted from these, so called, `Photospheres'.\\
This chapter describes how we used the Google Streetview and other image APIs to crawl images from famous sights in Berlin.

\subsection{Image Crawler}
We provide a Google Streetview image crawling python script inside the \texttt{streetviewcrawler} folder. This script takes a csv file as input, which specifies the parameters the crawler needs, such as position and viewing angle. Chapter \ref{csv_file} explains the contents of this file.\\
All requirements for the crawler are specified in the \texttt{requirements.txt} file and can be installed with the command \texttt{\$pip install -r requirements.txt}.
Given a location in longitude and latitude, the image crawler will automatically create jpeg-images from different viewing angles. These angles can either be specified or automatically generated by calculating the angle between two geo-coordinates. In the latter case, the script will generate images in five degree steps from 30 degrees to the left to 30 degrees to the right of the calculated viewing angle. If specified by hand, we use one degree steps from the starting to the end angle.\\
Because photospheres can change or get removed, the API uses the closest photosphere to the location provided. This might not be the location specified in the csv file

\subsection{Setup of viewing parameters}\label{csv_file}
Explain csv file

\subsection{State of Automation (i.e. taking one image per viewing angle)}

\subsection{Current Limitations}
Full automation not possible as of current street view API state; wrong latitude/longitude; Streetview Image API returns different images than JavaScript API (which is being used on google maps website)

\subsection{Possible Improvements}
Use Classifier to identify things in photsphere

\subsection{Google Places API/shutterstock/Flickr}